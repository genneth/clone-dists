\documentclass[10pt,UKenglish]{article}
\usepackage[T1]{fontenc}
\usepackage{lmodern}
\usepackage{fouriernc}
\usepackage[latin9]{inputenc}
\usepackage[a4paper]{geometry}
\geometry{verbose}
\pagestyle{plain}
\usepackage{babel}
\usepackage{graphicx}
\usepackage{amsmath}
\usepackage{setspace}
\onehalfspacing
\usepackage{siunitx}
\usepackage{microtype}
\usepackage{nicefrac}
\usepackage{subfigure}
\usepackage[unicode=true, pdfusetitle, bookmarks=true, bookmarksnumbered=false, bookmarksopen=false, breaklinks=false, pdfborder={0 0 1}, backref=false, colorlinks=false, pdfstartview=FitBH]{hyperref}
\usepackage[margins,adjustmargins]{trackchanges}
\addeditor{GZ}

\begin{document}

\title{Supplementary theory}
\author{Gen Zhang}
\date{}
 
\maketitle

\renewcommand{\thesection}{S-\Roman{section}}
\numberwithin{equation}{subsection}

The key feature of adult tissue is homeostasis in the absence of external
factors such as wounding or infection; in epithelia like the oesophageal lining,
this gives us two precise quantitative rules: the total number of cells in the
basal layer must be conserved, and the total number of cells in the whole tissue
must be conserved. This means that, for example, we must balance the loss of
cells from the basal layer with the generation of new cells; failure to do so
would lead to exponential growth (cancer) or loss (tissue failure). Although
this delicate balance must be maintained, it may be kept only on average ---
small fluctuations in cell number or density is not catastrophic.

We see this conservation of basal cells through clonal labelling: it is observed
that the average size of clones increase (main text, fig ??); at the same time,
\annote[GZ]{the density of clones decrease}{do we have a graph for this? is it
any good?}, such that the total number of labelled cells remains roughly
constant. The interpretation is that occasionally all cells in a clone will
detach from the basal layer, resulting in a zero-cell clone, from which it will
never recover. Thus, the remaining clones must expand (on average) to maintain
homoeostasis.

The central observation is that at late times the basal clone distribution
\emph{scales}: the probability of finding a 2 cell clone at 1 month is twice
that of finding a 4 cell clone at 2 months, which in turn is twice that for an 8
cell clone at 4 months, and so on. \annote[GZ]{Klein et al. 2007}{should I avoid
mentioning Clayton?} showed that this implies that \annote[GZ]{there is only one
process which controls the eventual clone evolution}{I took this line out of
Klein '07; is that a bad idea?}. This is the observation that suggests a simple
description of the tissue maintainence.

At this point, we must make assumptions to make progress. Clayton et al.
introduced a model of committed progenitors (main text, Fig ??) which can
explain the observations above. A key feature of this model is the existence of
progenitor cells which divide, and adopt one of three fates: produce a pair of
progenitors, produce a pair of differentiated cells, or produce one of each. It
is important to emphasize that we are not stating in particular when this choice
is made --- it may be independently in each daughter cell, or upon division
itself, or even prior to division. Furthermore, we classify cells purely based
on its \emph{fate}, rather than gene expression; progenitors are defined to be
cells which will divide, and differentiated otherwise. 

Here, we go further, and consider both the basal clone size distribution, and
the joint basal and suprabasal distribution. We find that it becomes possible to
infer, from data, the parameters of the model: the division rate of progenitors
in the basal layer, the stratification rate of differentiated cells out of the
basal layer, and the ratio of self-renewing divisions versus all divisions.
Furthermore, this enriched data set allows us to reduce the total number of
observations needed, and also to provide an independent validation that genetic
markers are reliable indicators of fate. 

Our methodology is based on Bayesian inference, the technical details of which
are in section ??. The process by which numerical predictions of the joint clone
size distributions are generated is explained in section ??; this is necessarily
mathematical, but is not essential for understanding the overall strategy.
Finally, we demonstrate an application to quantify drug action in section ??.

\section{Critical branching processes and scaling}

The process in figure ?? (main text) is an example of a \emph{branching process}, a well-studied family of processes with many known mathematical results (A\&N). We summarise some relevant results here.

Concretely, we consider the following continuous time process:
\begin{align}
A &\overset{\lambda}{\longmapsto} \begin{cases}
A+A & r \\
A+B & 1-2r \\
B+B & r\end{cases} & B &\overset{\gamma}{\longmapsto} C & C &\overset{\mu}{\longmapsto} \emptyset
\label{eq:really-full-model}
\end{align}
We use the labels $A$, $B$ and $C$ to respectively represent progenitors,
differentiated cells in the basal layer, and suprabasal cells which can still be
tracked. Asymptotically, as $t \rightarrow \infty$, the average clone size
(including zero-cell clones) will be constant, making this an example of a
3-type \emph{critical} branching process. Notice that in addition, if we ignore
suprabasal cells (C), we obtain another critical branching process describing
only the basal dynamics.

In general, critical branching processes (under mild conditions easily fulfilled
here) have some simple asymptotic properties:
\begin{itemize}
	\item Almost surely the clone will become extinction: $\lim_{t
	\rightarrow 0} p_0(t) \rightarrow 0$. Specifically, $1 - p_0(t) \sim
	1/t$. 
	\item At the same time, the average surviving clone size increases to
	maintain the same overall average: $\left\langle n^\textrm{surv.}
	\right\rangle \sim t$.
	\item The clone size distribution, conditioned on survival, scaled by
	the average surviving clone size, converges in distribution to an
	exponential: $$P\left(\frac{n}{\left\langle n^\textrm{surv.}
	\right\rangle} < s \middle| n > 0 \right) = 1 - e^{-s}.$$
\end{itemize}

\section{Bayesian inference of parameters}

\section{Computation of joint clone size distributions}

\section{ATRA, quantified}

\end{document}

