\documentclass[10pt,UKenglish]{article}
\usepackage[T1]{fontenc}
\usepackage{lmodern}
%\usepackage{fouriernc}
\usepackage[utf8]{inputenc}
\usepackage[a4paper]{geometry}
\geometry{verbose}
\pagestyle{plain}
\usepackage{babel}
\usepackage{graphicx}
\usepackage{amsmath}
\usepackage{esint}
\usepackage{setspace}
\onehalfspacing
%\usepackage{siunitx}
\usepackage{microtype}
%\usepackage{nicefrac}
%\usepackage{subfigure}
\usepackage[unicode=true, pdfusetitle, bookmarks=true, bookmarksnumbered=false, bookmarksopen=false, breaklinks=false, pdfborder={0 0 1}, backref=false, colorlinks=false, pdfstartview=FitBH]{hyperref}
\usepackage[margins,adjustmargins]{trackchanges}
\addeditor{GZ}

\begin{document}

\title{Supplementary theory}
\author{Gen Zhang}
\date{}
 
\maketitle

\renewcommand{\thesection}{S-\Roman{section}}
\numberwithin{equation}{subsection}

The key feature of adult tissue is homeostasis in the absence of external
factors such as wounding or infection; in epithelia like the oesophageal lining,
this gives us two precise quantitative rules: the total number of cells in the
basal layer must be conserved, and the total number of cells in the whole tissue
must be conserved. This means that, for example, we must balance the loss of
cells from the basal layer with the generation of new cells; failure to do so
would lead to exponential growth (cancer) or loss (tissue failure). Although
this delicate balance must be maintained, it may be kept only on average ---
small fluctuations in cell number or density is not catastrophic.

We see this conservation of basal cells through clonal labelling: it is observed
that the average size of clones increase (main text, fig ??); at the same time,
\annote[GZ]{the density of clones decrease}{do we have a graph for this? is it
any good?}, such that the total number of labelled cells remains roughly
constant. The interpretation is that occasionally all cells in a clone will
detach from the basal layer, resulting in a zero-cell clone, from which it will
never recover. Thus, the remaining clones must expand (on average) to maintain
homoeostasis.

The central observation is that at late times the basal clone distribution
\emph{scales}: the probability of finding a 2 cell clone at 1 month is twice
that of finding a 4 cell clone at 2 months, which in turn is twice that for an 8
cell clone at 4 months, and so on. Klein et al. 2007 showed that this implies
that \annote[GZ]{there is only one process which controls the eventual clone
evolution}{I took this line out of Klein '07; is that a bad idea?}. This is the
observation that suggests a simple description of the tissue maintainence.

At this point, we must make assumptions to make progress. \annote[GZ]{Clayton et
al.}{should I avoid mentioning Clayton?} introduced a model of committed
progenitors (main text, Fig ??) which can explain the observations above. A key
feature of this model is the existence of progenitor cells which divide, and
adopt one of three fates: produce a pair of progenitors, produce a pair of
differentiated cells, or produce one of each. It is important to emphasize that
we are not stating in particular when this choice is made --- it may be
independently in each daughter cell, or upon division itself, or even prior to
division. Furthermore, we classify cells purely based on its \emph{fate}, rather
than gene expression; progenitors are defined to be cells which will divide, and
differentiated otherwise. 

Here, we go further, and consider both the basal clone size distribution, and
the joint basal and suprabasal distribution. We find that it becomes possible to
infer, from data, the parameters of the model: the division rate of progenitors
in the basal layer, the stratification rate of differentiated cells out of the
basal layer, and the ratio of self-renewing divisions versus all divisions.
Furthermore, this enriched data set allows us to reduce the total number of
observations needed, and also to provide an independent validation that genetic
markers are reliable indicators of fate. 

Our methodology is based on Bayesian inference, the technical details of which
are in section ??. The process by which numerical predictions of the joint clone
size distributions are generated is explained in section ??; this is necessarily
mathematical, but is not essential for understanding the overall strategy.
Finally, we demonstrate an application to quantify drug action in section ??.

\section{Critical branching processes and scaling}

The process in figure ?? (main text) is an example of a \emph{branching
process}, a well-studied family of processes with many known mathematical
results (Atherya and Ney). We summarise some relevant results here.

Concretely, we consider the following continuous time process:
\begin{align*}
A &\overset{\lambda}{\longmapsto} \begin{cases}
A+A & r \\
A+B & 1-2r \\
B+B & r\end{cases} & B &\overset{\gamma}{\longmapsto} C & C &\overset{\mu}{\longmapsto} \emptyset
\end{align*}
We use the labels $A$, $B$ and $C$ to respectively represent progenitors,
differentiated cells in the basal layer, and suprabasal cells which can still be
tracked. All processes are assumed to be Poisson, with rate rates $\lambda$, $\gamma$ and $\mu$; equivalently, we may say that individual cells have lifetimes distributed exponentially.

Asymptotically, as $t \rightarrow \infty$, the average clone size (including
zero-cell clones) will be constant, making this an example of a 3-type
\emph{critical} branching process. Notice that in addition, if we ignore
suprabasal cells (C), we obtain another critical branching process (this time
2-type) describing only the basal dynamics.

In general, multi-type critical branching processes (under mild conditions
easily fulfilled here) have some simple asymptotic properties\footnote{The
notation $x \sim y$ means that in the limit under consideration, $x/y
\rightarrow C$ for some finite constant $C$.}:
\begin{itemize}
	\item Almost surely the clone will become extinction: $\lim_{t
	\rightarrow 0} p_0(t) \rightarrow 0$. Specifically, $1 - p_0(t) \sim
	1/t$. 
	\item At the same time, the average surviving clone size increases to
	maintain the same overall average: $\left\langle n^\textrm{surv.}
	\right\rangle \sim t$.
	\item The clone size distribution, conditioned on survival, scaled by
	the average surviving clone size, converges in distribution to an
	exponential: $$\lim_{t \rightarrow \infty} P\left(\frac{n}{\left\langle
	n^\textrm{surv.} \right\rangle} < s \middle| n > 0 \right) = 1 -
	e^{-s}.$$ This is exactly the statement of scaling.
	\item Asymptotically, the ratios of the types of cells in a clone almost
	surely approaches the ratios of averages, which is given by a
	consideration of detailed balance in equilibrium. For example, for our
	process above, we must have $\lambda \left\langle n_A \right\rangle -
	\gamma \left\langle n_B \right\rangle = 0$ and $\gamma \left\langle n_B
	\right\rangle - \mu \left\langle n_C \right\rangle = 0$.
\end{itemize}

From the point of view of inferring the exact kinects of tissue maintainence,
this universal behaviour of all critical branching processes is unfortunate.
Indeed, from the late time data, the only measurable parameters are the overall
time scale $\tau$ of the growth in average surviving clone size (the slope of
figure ??); and assuming that the cell types can be unambiguously identified,
the ratios of the rate constants (via the equilibrium ratios of cell types). As
such, we need to confront early time data, away from the asymptotic regime,
where such differences can be manifest.

%As a final piece of generality, we consider an age-dependent branching process;
%that is, cell lifetimes are no longer exponential. It can be shown (Atherya and
%Ney) that the asymptotic behaviour is intact, again subject to mild conditions
%easily fulfilled in biologically relevant models. Klein et al. 2007 showed that
%whilst in principle one could see significant deviations from the age-indepedent
%process, it requires inducing cells at precisely the same point in their cell
%cycle. In practise, the induction protocol is incapable of synchronising the
%labelled cells to any great extent, and the resulting clone size distribution
%does not differ noticeably from the age-independent process. For this reason, we
%will work solely with the age-indepedent process, for computational
%tractability.

\section{Inference of parameters}

Our model has 4 separate parameters: $r$, $\lambda$, $\gamma$ and $\mu$.
Defining $\rho$ to be the proportion of progenitors in the basal layer, the
homoeostasis requirement gives $$\lambda \rho = \gamma (1-\rho).$$ Similarly,
homoeostasis gives a condition for the number of suprabasal cells per basal cell
$m$: $$\gamma (1-\rho) = \mu m.$$ Notice that with these relationships, we can
eliminate $\gamma$ and $\mu$ in favour of $\rho$ and $\mu$. In addition, these
are related to clonal composition in the asymptotic limit by detailed balance
arguments mentioned in section ??:
\begin{align*}
\rho &= \lim_{t \rightarrow \infty}\frac{\left\langle n_A \right\rangle}{\left\langle n_A + n_B\right\rangle} \\
m &= \lim_{t \rightarrow \infty}\frac{\left\langle n_C \right\rangle}{\left\langle n_A +n_B\right\rangle}
\end{align*}
As such we can measure $m$ directly \annote[GZ]{(table ??)}{Phil hard work at
counting results in a table!}, but since it is not possible to visually
distinguish progenitors from differentiated cells in the basal layer, $\rho$ is
not directly measureable.

The remaining three parameters we infer directly from the observed clone size
distributions, using Bayes' Theorem. \note[GZ]{finish this; mostly just waffle
about the need to condition the data}

\section{Computation of joint clone size distributions}

In using Bayes' theorem we must repeatedly compute the likelihood of the
observed data for different parameters with reasonable accuracy. Because the
distribution we wish to calculate has small tails (exponentially decaying), it
is expensive to sample with a Monte-Carlo simulation, which would need vast
number of trials. At the same time, due to loss of cells, in a Markov treatment
of the system the transition matrix does not decompose into any helpful block
form, and so no truncation is safe. Instead, we proceed via a generating
function.

Let $p_{m,n,l}(t)$ be the probability of finding a clone with $n$ progenitors,
$m$ differentiated basal cells, and $l$ suprabasal cells. At time $t=0$, the
only non-zero element is $p_{1,0,0} = 1$. We define the generating function
$$F(x,y,z,t) = \sum_{m,n,l} x^m y^n z^l p_{m,n,l}(t).$$ This generating function
then proceeds to evolve according to a partial differential equation $$F_t =
\lambda \left[r x^2 + (1-2r) x y + y^2 - x \right] F_x + \gamma \left(z - y
\right) F_y + \mu \left(1-z \right) F_z,$$ where subscripts indicate
differentiation with respect to that variable and initial conditions
$F(x,y,z,0)=x$. Finally, noting that the equation is hyperbolic and of first
order, we can use the method of characteristics to get a system of non-linear
ordinary differential equation:
\begin{align*}
\dot x &= \lambda \left[r x^2 + (1-2r) x y + y^2 - x \right] \\
\dot y &= \gamma \left(z - y \right) \\
\dot z &= \mu \left(1-z \right)
\end{align*}
The solution to the generating function is then $F(x_0,y_0,z_0,t) = x$ with
$x(0) = x_0$, $y(0) = y_0$ and $z(0) = z_0$. In practice, one can use a
numerical integrator to find the solution to the above with great accuracy.

To perform the inverse transform to find $p_{m,n,l}$, note that $F$ is a series which is monotonic and finite in $x$, $y$ and $z$ on unit intervals. Thus it must be holomorphic on the unit disc in each of those variables. This then suggests a method of inversion based on contour integrals:
\begin{align*}
p_{m,n,l}(t) = \left(\frac{1}{2\pi i}\right)^3 \oint_{\mathcal{C}^3} \frac{F(x,y,z,t)}{x^{m+1} y^{n+1} z^{l+1}} dx\,dy\,dz
\end{align*}
where the contour $\mathcal{C}^3$ is an oriented volume defined by $|x|=|y|=|z|=1$, where the intent is to circle the respective origins of the complex planes in a counter-clockwise sense. Discretising the integral for a computer fortuitously yields a three-dimensional discrete Fourier transform.

To make contact with the data, we need to combine the counts for both types of basal cells. This can be done by defining $G(s,z) = F(s,s,z)$, which one may invert by a simpler 2-dimensional integral, with much reduced computational cost.

\section{ATRA, quantified}

\end{document}

