\documentclass[10pt,UKenglish]{article}
\usepackage[T1]{fontenc}
\usepackage{lmodern}
\usepackage{fouriernc}
\usepackage[latin9]{inputenc}
\usepackage[a4paper]{geometry}
\geometry{verbose}
\pagestyle{plain}
\usepackage{babel}
\usepackage{graphicx}
\usepackage{amsmath}
\usepackage{setspace}
\onehalfspacing
\usepackage{siunitx}
\usepackage{microtype}
\usepackage{nicefrac}
\usepackage{subfigure}
\usepackage[unicode=true, pdfusetitle, bookmarks=true, bookmarksnumbered=false, bookmarksopen=false, breaklinks=false, pdfborder={0 0 1}, backref=false, colorlinks=false]{hyperref}

\begin{document}

\title{Supplementary theory}
\author{Gen Zhang}
 
\maketitle

\renewcommand{\thesection}{S-\Roman{section}}
\numberwithin{equation}{subsection}

\section{Bayesian analysis of clone fate data}

The stochastic fate model 
\begin{align}
A &\overset{\lambda}{\longmapsto} \begin{cases}
AA & r \\
AB & 1-2r \\
BB & r\end{cases}, & B &\overset{\gamma}{\longmapsto} \emptyset,
\label{eq:basal-model}
\end{align}
has three fundamental parameters. The cycling rate $\lambda$ may be directly measured from observation of cycling rate; the proportion of proliferating progenitor cells $\rho$ may be measured from Ki67 expression; the time scale $\tau = \rho/r\lambda$ is directly measurable from the measured clone size distribution, which then gives the symmetric division rate $r$. However, if we can obtain fate data \emph{which also tracks suprabasal cells}, then it is possible to directly determine all parameter from the clone size distribution. Simply fitting the parameters to a given set of data can be difficult, and fail to fully express the confidence in the fitted parameters. As such, we proceed more carefully and fully via Bayesian analysis. In addition, such an analysis would be independent of any biological assumptions such as whether Ki67 is a reliable marker for proliferation.

\subsection{Inferring parameters from observations}

Bayes' theorem gives the \emph{posterior distribution} of the parameters $\theta$ as a result of some observations $O$: $$p(\theta|O) = \frac{p(O|\theta)}{p(O)} p(\theta).$$ The factor of $p(O)$ acts as a \emph{normalisation} to make the left-hand side a proper probability distribution and so is biologically uninteresting. 

The \emph{likelihood} $p(O|\theta)$ depends fundamentally on the predicted clone size distributions $P_{mn}$ (see below, section \ref{sec:p-mn-calculation}). The observations $O$ are a set of cohorts $O_t = \{(m,n)\}$ with $m$ basal cells and $n$ suprabasal cells seen at a time $t$ after induction. We assume that the induction probability is sufficiently low that the initial conditions are essentially just single cell clones. We can filter out induction of suprabasal cells by conditioning $P_{mn}$ on $m \ge 1$, i.e. \emph{floating clones}. Even so, we have an unknown chance of inducing a terminally differentiated cell in the basal layer; we do not want to make extraneous assumptions about induction rates, and filter out such data by conditioning on clones of size 2 or above, i.e. $m+n \ge 2$. The latter criterion also excludes from consideration extinct clones, which are by definition unobservable. As such, we define the \emph{observable clone distribution} $$P^\textrm{obs.}_{mn} = \frac{P_{mn}}{1 - P_{00} - P_{10} - \sum_j P_{0j}}.$$ Now given that the observation $O_t$ is filtered as above, we can compute the likelihood: $$p(O_t|\theta) = \prod_{(m,n) \in O_t} P^\textrm{obs.}_{mn}(t).$$

The only ingredient remaining is the \emph{prior distribution} $p(\theta)$. To be completely rigorous, we should pick it to be a maximum entropy distribution over the parameter space $\theta$ <jaynes>. In this case that means a uniform distribution in $\rho$ and $r$, and (an improper distribution) log-uniform in $\lambda$: $p(\theta) \propto 1/\lambda$. <maybe the following sentences are redundant> However, we will be working with observation sets of several hundred, which makes the likelihood sharply peaked in a small region, and unless the prior changes significantly over this region, it does not really matter what prior is chosen. For this work, uniform distributions over some bounded region was chosen.

Multiple time points can be combined by simply multiplying the likelihood functions together, since by the very nature of cohort studies there are no correlations between different observations at different times.

<translucent blobs> shows a typical example of the inference. The posterior distribution is well-localised distribution, resembling a rugby ball. <defer discussion of what the probabilities mean until later>

\subsection{\label{sec:p-mn-calculation}Clone size distributions with suprabasal cells}

The central prediction of the theory is a distribution $P_{mn}(t)$, the probability of a clone having $m$ basal cells and $n$ suprabasal cells at time $t$ after induction. Since we are tracking suprabasal cells, we will modify the stochastic model \ref{eq:basal-model}:
\begin{equation*}
B \overset{\gamma}{\longmapsto} C,
\end{equation*}
where suprabasal cells are represented as $C$.

Whilst we cannot experimentally distinguish between proliferating CP cells ($A$) and terminally differentiated cells in the basal layer ($B$), it is necessary to do so mathematically as they produce entirely different progeny trees. Thus we will work with the finer grained distribution $P_{m_A m_B n}$ which is related $$P_{mn} = \sum_{m_A + m_B = m} P_{m_A m_B n}.$$ Since this model is Markovian, we can write down the master equation <> which is an infinite set of first order differential equations: 
\begin{equation}
\frac{dP_{m_A m_B n}}{dt} = \sum_{m_A^\prime m_B^\prime n^\prime} T_{m_A m_B n; m_A^\prime m_B^\prime n^\prime} P_{m_A^\prime m_B^\prime n^\prime}. \label{eq:infinite-master}
\end{equation}

The model <> has a special property: the number of cells in a clone can never decrease. This means that $P_{m_A m_B n}(t)$ can only depend on the probabilities of clones with the same or fewer total cells at earlier times. Thus if we only care about clones up to a certain size we can truncate the infinite set $P_{m_A m_B n}$ to a finite one, and package them up into a vector $\mathbf P$. At the same time, only a finite number of elements of $T$ will be needed, and moving them into a matrix $\mathbf T$ allows equation \eqref{eq:infinite-master} to be written compactly as $$\frac{d\mathbf P}{dt} = \mathbf{T P}.$$ This is then a standard differential equation, whose solution is a matrix exponential $$\mathbf P = \exp(\mathbf T t) \mathbf P_0,$$ where $\mathbf P_0$ is the initial condition, i.e. one type $A$ cell.

\subsection{\label{sec:ball-plane}Consistency and improvement with basal statistics}

The need to count suprabasal cells limits the scope of the experiment, as eventually shedding becomes a serious effect, and its statistical effects are likely to be environmentally dominated. Counting basal statistics only, it becomes possible to extend the experiment to longer time scales, and provides both a consistency check of the methodology and also improves the inference accuracy. Whilst it would be straightforward to extract $\tau = \rho/r\lambda$ as shown in <>, it is also worthwhile to run a full-scale Bayesian analysis.

Fundamentally, there is very little change in the procedure described above. The main differences are:

\begin{itemize}
\item The branching model <> does not enjoy the triangular structure which allowed the probabilities to be computed simply via a matrix exponential, and the full infinite system must be treated. The basal distributions $P_m$ are best computed via generating function methods, e.g. <tedious paper; maybe there should be a section here to explain how to do it?>. It is also possible to do a simple truncation, but the errors are difficult to bound. Either way, it is possible to calculate the likelihood function $p(O|\theta)$.
\item The posterior distribution should now be expected to be a hyperbolic sheet in $(\rho, r, \lambda)$ space, along some definite $\tau = \rho/r\lambda$. Since this distribution is not particularly well localised, the true prior distribution should be used. Specifically, one which is log-uniform in $\lambda$.
\end{itemize}

Consistency between basal and basal-plus-suprabasal statistics manifest as non-trivial overlaps of significant parts of the respective posterior distributions (see figure <>). These estimates may then be combined by the usual multiplication of likelihoods (figure <>, main text).

\subsection{Meaning of posterior probabilities}

It is important to be clear about what the 95\% confidence region in figure <> means. The exact statement is that \emph{given the correctness of the underlying model, there is only a 1 in 20 chance that the data collected is perverse to the extent that the true parameters of the model lie outside of the region}.

<inference of MC-generated data; shows spread in inferred ball locations>

Although the posterior distribution is nicely peaked, it would be a mistake to think that the true parameters are more likely to be at the most probable value, as opposed to some definite distance away from the peak. This can be clearly seen in figure <> above, where the true parameters are actually never very close to the centres of the confidence regions. This can be understood very simply as a consequence of multi-dimensional distributions. For example, if we had a three-dimensional Gaussian $$p(\mathbf{r}) \propto \exp\left(-|\mathbf r|^2\right),$$ the expectation $\left\langle \mathbf |\mathbf r| \right\rangle > 0$ and is in fact of the order of the standard deviation. In section \ref{sec:ball-plane}, when we look for consistency against long time course basal statistics, we should not be surprised that the hyperbolic sheet does not exactly intersect the centre of the ``rugby ball''.

<need to be more focused>

\section{Impact of stem cell population}

The existence of a long-lived, slow-cycling stem population has the potential to change the clone size distribution. Indeed one of the most significant traits is the relative enrichment of single cell clones --- interpreted as a single stem cell which was induced but then never divided <main text>. Here, we show that deviations from the balanced CP model (\ref{eq:basal-model}) are small, and essentially constrained to a very slight increase in small clones, but well within currently achievable statistical errors.

Fundamentally, if stem cells can differentiate to committed progenitors, then the CP dynamics must be slightly imbalanced --- otherwise homoeostasis would be violated. At the same time, for the stem cells to be only slowly cycling, the asymmetry must be very small. Figure <main text> shows that the balanced CP process is an excellent fit to the data, therefore, any deviation must be very small. This would be all consistent with the hypothesis that stem cells are recruited for repair <wounding cite>, but are otherwise quiescent in normal tissue.

In homoeostasis, stem cells do not appear to divide symmetrically (figure <main text>), so we posit a simple division process: 
\begin{equation}
S \overset{\Lambda}{\longmapsto} S+A, \label{eq:stem-division-model}
\end{equation} where the division is assumed to occur randomly. This is a mathematical simplification and, to within accessible statistics, is exact in the limit of small numbers of active stem cells. It does not imply a biological assumption of stochasticity; in particular, we are not ruling out other dynamics of the stem population, which must necessarily include self-renewal and environmentally controlled division to produce the patterning observed (figure <main text>). It is simply the case that cohort statistics is dominated by the CP sector, and therefore we are largely measuring the effects of the stem population on the CP sector. Furthermore, we simplify the CP dynamics to be
\begin{equation}
A \overset{\lambda}{\longmapsto} \begin{cases}
AA & r - \Delta/2 \\
A & 1 - 2r \\
\emptyset & r + \Delta/2,
\end{cases}\label{eq:subcritical-cp-model}
\end{equation} which allows a rigorous treatment. The basal statistics can be recovered by noting that $B$ population is entirely slave to $A$ cells, and can be approximated (to very good accuracy, see <tedium>) by scaling the CP clone size by $1/\rho$.

Due to the low induction frequency and low proportion of cycling stem cells, we expect each clone contain at most one stem cell. Thus we have two cases to consider: clones with no stem cells at all, which will all eventually go extinct, and clones with one stem cell which will equilibrate at some finite clone size. Below, we take these cases separately.

\subsection{Subcritical CP dynamics}

Model \ref{eq:subcritical-cp-model} is a \emph{subcritical branching process} <cite>. The expected clone size decreases monotonically $\langle n \rangle = n_0 e^{-\Delta \lambda t}.$ Define $\delta = \Delta/2r$ and 
\begin{align*}
\alpha &= \frac{2\delta}{\left(1 - e^{-\Delta \lambda t}\right)\left(1-\delta\right)}, \\
\beta &= \frac{\alpha}{1-\alpha}.
\end{align*}
The clone size distribution is:
\begin{equation*}
P_m(t) = \begin{cases}
1 - \beta e^{-\Delta \lambda t} & m=0 \\
\frac{\beta^2}{(2\beta +1)(\beta +1)}\left(1-\beta\right)^m & m\ge1
\end{cases}
\end{equation*}
Note that the distribution is geometric for $m \ge 1$. This information can be packaged up into a single \emph{generating function} $f(z,t) = \sum_m P_m(t) z^m$:
\begin{equation*}
f(z,t) = \frac{e^{\Delta \lambda t}\left[(1-z) + (1+z)\delta\right] + (z-1)(1+\delta)}{e^{\Delta \lambda t}\left[(1-z) + (1+z)\delta\right] + (z-1)(1-\delta)}.
\end{equation*}

\subsection{\label{sec:subcritical-immigration}Stem supported clones}

Homoeostasis requires that the loss of CP cells be balanced by creation via stem cell division. If the overall cycling stem population as a proportion of all basal cells is $\rho_S$, then we must have $$\rho_S \Lambda = (1-\rho_S) \Delta \lambda.$$ The division process \ref{eq:stem-division-model} preserves the number of stem cells and each of the created CP cells go on to divide and differentiate independently.

Consider then a stem cell which undergoes asymmetric division $n$ times within a time $t$, at times $t_1 < t_2 < \ldots < t_n$. Since the process is Poisson, the probability of having $n$ divisions is independent of when they occur, and is $\Lambda^n e^{-\Lambda t}$. The generating function for the distribution of CP cells for such a sequence of divisions is then
\begin{align*}
g_n(z,t) &= \Lambda^n e^{-\Lambda t} \int_0^t dt_1 \int_{t_1}^t dt_2 \ldots \int_{t_{n-1}}^t dt_n f(z,t-t_1) \times f(z,t-t_2) \times \ldots \times f(z,t-t_n) \\
 &= \frac{\Lambda^n e^{-\Lambda t}}{n!} \left[\int_0^t f(z,t-\tau) d\tau \right]^n
\end{align*}
where we notice that the need to order the times $t_n$ can be removed since it just results in an over counting. The generating function for the number of CP cells in total is then
\begin{align*}
g(z,t) &= \sum_n g_n(z,t) \\
 &= e^{-\Lambda t} \exp\left[\Lambda \int_0^t f(z,t-\tau) d\tau \right] \\
 &= \exp\left\{\Lambda \int_0^t \left[f(z,\tau) - 1\right] d\tau \right\} \\
 &= \exp\left(\frac{2 \Delta \Lambda t}{1-\delta}\right) \left\{ \frac{\left(e^{\Delta \lambda t}-1 \right)(z-1)}{2\delta} + \frac{1}{2}\left[(1-z) + e^{\Delta \lambda t}(1+z) \right]\right\}^{-2\delta\Lambda/\lambda}.
\end{align*}

The distribution $P_m$ may be obtained by computing the coefficients of the Taylor expansion $$g(z,t) = \sum_m P_m(t) z^m.$$ In this case, exact expressions are not easily available, but is still numerically efficient to extract. Note that since the function $g(z,t)$ has a branch point in the complex plane, the radius of convergence of the Taylor expansion has a finite radius of convergence $R$. Since $$g(z=1,t) = \sum_m P_m(t) = 1,$$ the radius of converge $R \ge 1$. The ratio test for convergence gives $$\lim_{m\rightarrow \infty} \left| \frac{P_{m+1}}{P_m} \right| = \frac{1}{R},$$ thus for large $m$, the distribution becomes geometric.

\subsection{Mean clone size}

Since the imbalance $\Delta$ is very small, we should expect that at short times the dynamics is essentially the same as the balanced case. Further, we expect that major deviations only occur on time scales $1/\Lambda \sim 1/\Delta \lambda$.

Experimentally, we induce single cells. These may be either a CP or a stem cell. Assuming proportional induction, the generating function for the total distribution is $$h(z,t) = \rho_S z g(z,t) + (1-\rho_S) f(z,t).$$ The factor of $z$ simply accounts for the fact that each stem-derived clone will contain one stem cell. The average clone size, by construction, is constant. The observed average size will still increase, since extinct clones will not contribute: 
\begin{align*}
\left\langle n^\textrm{obs} \right\rangle &= \frac{1}{1 - h(z=0)} \\
  &= \frac{1}{\rho_S + \beta (1-\rho_S) e^{-\Delta \lambda t}}.
\end{align*}
At early times, we have the linear growth characteristic of a critical branching process, before reaching a plateau at $1/\rho_S$ as $t\rightarrow\infty$. The turn over occurs, as expected, around $1/\Delta\lambda$. Since the long time basal data, out to a year, does not show any sign of a plateau, we can infer an upper bound for imbalance $\Delta$.

<figure showing difference>

\subsection{Basal distribution}

Given the generating function <ref>, we can numerically extract the clone size distribution. As we shall see, the existence of a stem cell compartment is still consistent with the experimental observations.

Heuristically, we can understand why there is almost no change. As mentioned in section \ref{sec:subcritical-immigration}, the distribution for stem-supported clones has a exponential tail. This remains true for the equilibrium distribution. Upon scaling, this tail dominates the observable clones; deviations from this exponential distribution is only visible in small clones. Furthermore, this exponential tail is weaker than the tail from the near critical CP sector, and so is drowned out.

<figure show that the stem compartment makes no difference to the clone size distributions; maybe a figure to show the final equilibrium distributions?>

\end{document}
